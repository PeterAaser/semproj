\documentclass[journal]{IEEEtran}
\usepackage{blindtext}
\usepackage{graphicx}

\usepackage[cmex10]{amsmath}
\usepackage{array}

\usepackage{mdwmath}
\usepackage{mdwtab}

\usepackage{fixltx2e}

% for showing geometry details wrt pdf rendering
%\usepackage[showframe]{geometry}% http://ctan.org/pkg/geometry

% correct bad hyphenation here
\hyphenation{op-tical net-works semi-conduc-tor}


\begin{document}

% can use linebreaks \\ within to get better formatting as desired
\title{Investigating in-vitro neuron cultures as computational reservoir}

\author{Peter Aaser}
%
% make the title area
\maketitle


\begin{abstract}
  ITT: memes\\
  underfull hbox badness $10000$
  It's an abstract kind of shitpost
\end{abstract}


\begin{IEEEkeywords}
memes, reddit, unfunny, low-energy
\end{IEEEkeywords}

\section{Introduction}
Reservoir Computing is a novel approach to machine learning, using reservoirs
containing dynamic complex systems as computational elements. 
\cite{schrauwen_overview_2007}
Many different systems have been used as reservoirs, ranging from
virtual reservoirs such as random boolean networks and recurrent neural nets,
to real physical systems such as nano-carbon tubes.
Building on previous work we investigate using in-vitro neuron cultures as
reservoirs, describing the necessary infrastructure for performing such
experiments with live neurons.

\section{Background}
\subsection{Reservoir computing}
A big disconnect between machine learning and how we perceive human reasoning is
how to solve and encode problems of temporal nature.
While feed forward structures which can only consider their current input can be
extended to reduce temporal problems into structural ones by adding delays
\cite{schrauwen_overview_2007}, a more natural approach is to use
recurrent structures that modify themselves in response to input.
A widely studied example of this approach is the recurrent neural network,
however the increase in expressiveness makes these structures very hard to train\cite{bertschinger_real-time_2004}.
Another example are random boolean networks \cite{gershenson_introduction_2004} studied by Kauffman as a
model for genetic regulatory networks.
Common for these two systems, and many more such as cellular
automata \cite{sipper_emergence_1999} and liquid state machines exhibit the
properties of \textit{complex systems} \cite{langton_computation_1990}.
Complex systems are systems that are, in a sense, magic.
In order to harness the power of these complex systems it is fruitless to
attempt to shape the topology and dynamics of the systems towards some specific
goal. Instead, the complex systems are used as \textit{reservoirs} which we will
interact with using a simple linearly separable input and output processing
layer which can be easily trained. \cite{schrauwen_overview_2007}
TO-DO: back linearly separable claims.
\subsection{Neurons}
\begin{figure*}[p]
    \centering
    \includegraphics[width=\textwidth]{images/"Neuron anatomy".png}
    \caption{a neuron}
    \label{fig:neuron_anatomy}
\end{figure*}
Neurons are vastly complex entities, communicating through complex electric
and chemical signals. However, since we are more interested in the emergent
properties of neurons in the context of reservoir computing a superficial
description suffices.
We will only consider a generalized version of the neuron, but in our
experiments a plethora of different neurons are used, although they
all share the basic similarities described here.
The anatomy of a neuron is shown in \ref{fig:neuron_anatomy} and can roughly be
divided into the following parts:
\subsubsection{Soma}
The main body of the neuron. While we will view neurons as simple network nodes
it is important to note that the neuron is highly complex, it can blah blah
\subsubsection{Dendrites}
To sense its surroundings the neuron is equipped with dendrites. These
branching structures act as receivers, propagating electro-chemical stimuli to
the cell body. Their reach is only to the immediate vicinity of the cell, they
do not form longer connections.
\subsubsection{Axon}
The axon is a long tendril, extending over a meter in the case of the sciatic
nerve TO-DO maybe embed link? which transmits information as electrical pulses
to other neurons. An axon can branch off and reach multiple neurons, it is not a
one to one connection.
\subsection{MEA2100}
To perform experiments a MEA2100 system has been purchased from multichannel systems.
The MEA2100 system is built to conduct experiments on in-vitro cell cultures, 
specifically (but not exclusively) neurons.
The principal components of the MEA2100 systems are:
\subsubsection{Micro electrode array}
To experiment on the properties of cells or other electrically active subjects the
micro electrode array (MEA) is used. As the name implies the MEA is equipped with
an array of electrodes able to both measure the electrical properties of the 
experiment subjects, as well as applying outside stimuli, acting in a sense as
output and input for the subject.
\subsubsection{Headstage}
The electrodes of the MEAs are measured and stimulated by the headstage which
contains the necessary high precision electronics needed for microvolt range readings.

\subsubsection{Interface board}
The interface board connects to up to two head-stages and is responsible for interfacing
with the data acquisition computer, as well as auxiliary equipment such as temperature
controls.
The interface board processes and filters data from the head-stages which can then be
acquired on a normal computer connected via USB. Additionally the interface board 
contains a Texas instruments TMS320C6454 digital signal processor which can optionally
be programmed by users of the system and allows a second process to interface via a 
secondary USB.

\section{Methodology}
Kind of a misnomer, really

\section{Results}
Yeah, about that...

\section{Conclusion}
In short you talk like a fag and your shit's all retarded. Thanks

\section*{Acknowledgment}
The authors would like to thank...\\
Sandoz, couldn't have done it without you.


\bibliography{mylib} 
\bibliographystyle{ieeetr}

\end{document}
