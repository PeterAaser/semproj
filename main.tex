\documentclass[journal]{IEEEtran}
\usepackage{blindtext}
\usepackage{graphicx}

\usepackage[cmex10]{amsmath}
\usepackage{array}

\usepackage{mdwmath}
\usepackage{mdwtab}

\usepackage{fixltx2e}


% correct bad hyphenation here
\hyphenation{op-tical net-works semi-conduc-tor}


\begin{document}

% can use linebreaks \\ within to get better formatting as desired
\title{Investigating in-vitro neuron cultures as computational reservoir}

\author{Peter Aaser}
%
% make the title area
\maketitle


\begin{abstract}
  ITT: memes\\
  underfull hbox badness $10000$
  It's an abstract kind of shitpost
\end{abstract}


\begin{IEEEkeywords}
memes, reddit, unfunny, low-energy
\end{IEEEkeywords}

\section{Introduction}
Reservoir Computing is a novel approach to machine learning, using reservoirs
containing dynamic complex systems as computational elements. 
\cite{schrauwen_overview_2007}
Many different systems have been used as reservoirs, ranging from
virtual reservoirs such as random boolean networks and recurrent neural nets,
to real physical systems such as nano-carbon tubes.
Building on previous work we investigate using in-vitro neuron cultures as
reservoirs, describing the necessary infrastructure for performing such
experiments with live neurons.

\section{Background}
\subsection{Reservoir computing}
A big disconnect between machine learning and how we perceive human reasoning is
how to solve and encode problems of temporal nature.
While feed forward structures which can only consider their current input can be
extended to reduce temporal problems into structural ones by adding delays
\cite{schrauwen_overview_2007}, a more natural approach is to use
recurrent structures that modify themselves in response to input.
A widely studied example of this approach is the recurrent neural network,
however the increase in expressiveness makes these structures very hard to train\cite{bertschinger_real-time_2004}.
Another example are random boolean networks #TODO CITE studied by Kauffman as a
model for gene activation.
such as in the human brain, or the organization of cells in ???.
More generally, a reservoir is a complex nonlinear system with irreducible properties
which we attempt to exploit in order to solve difficult problems such as classification.

Linearly separable input output, detecting spikes vs more abstract spike encoding


Alternative reservoirs
Describe more recent reservoirs


Post and preprocessing
This section should motivate why data is post and preprocessed wrt reservoir computing.
An important point is to show that these layers have no history, thus showing that a 
reservoir can help the agent to do some task it would be unable to perform with a linear
"uninteresting" reservoir. (even though a source of randomness would probably be enough 
for wall avoidance which should probably be emphasized).
\subsection{Neurons}
\subsection{Hardware/MCS}

\section{Methodology}
Kind of a misnomer, really

\section{Results}
Yeah, about that...

\section{Conclusiadore}
In short you talk like a fag and your shit's all retarded. Thanks


\appendices
\section{Proof of the First Zonklar Equation}
Some text for the appendix.
\cite{li_application_2015}

% use section* for acknowledgement
\section*{Acknowledgment}


The authors would like to thank...


% Can use something like this to put references on a page
% by themselves when using endfloat and the captionsoff option.
\ifCLASSOPTIONcaptionsoff
  \newpage
\fi


\bibliography{mylib} 
\bibliographystyle{ieeetr}

\end{document}
