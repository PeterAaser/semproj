\documentclass[journal]{IEEEtran}
\usepackage{blindtext}
\usepackage{graphicx}

\usepackage[cmex10]{amsmath}
\usepackage{array}

\usepackage{float}

\usepackage{mdwmath}
\usepackage{mdwtab}

\usepackage{fixltx2e}

\usepackage{tikz}
\usetikzlibrary{
  arrows,
  shapes,
  trees,
  intersections,
  shadows,
  positioning,
  decorations.pathmorphing,
  fit,
  petri,
  backgrounds,
  calc
}

% for showing geometry details wrt pdf rendering
%\usepackage[showframe]{geometry}% http://ctan.org/pkg/geometry

% correct bad hyphenation here
\hyphenation{op-tical net-works semi-conduc-tor}

\begin{document}

% can use linebreaks \\ within to get better formatting as desired
\title{Investigating in-vitro neuron cultures as computational reservoir}

\author{Peter Aaser}
%
% make the title area
\maketitle

\begin{abstract}
  The human brain acts as a vastly parallel, self organizing computer far
  surpassing silicon processors in robustness, plasticity and energy
  efficiency.
  To gain an understanding into how the brain works a cyborg (short for
  cybernetic organism) is currently being made in an interdisciplinary effort
  known as the \textit{NTNU cyborg project}.
  In this paper we explore applying a relatively novel machine learning
  technique, \textit{reservoir computing}, to harness the computational power of
  living neural networks.
  These living cultures of neurons have been successfully grown at the
  department of neuromedicine hospital in \textit{Micro Electrode Arrays} which
  allow them to be interfaced with using electrical stimuli and readouts.
  Using the theoretical foundation of reservoir computing these \textit{In
    Vitro} neural networks are used to control a simple agent in a simulation on
  a remote computer.
  Finally we suggest an extension to the idea of cybernetic organisms, extending
  the idea of a cyborg as a hybrid organism between the biological and mechanic
  into \textit{RC cyborg}, a robot which utilizes reservoir computing.
  To this end a software framework is described, facilitating the use of several
  different reservoirs in addition to neural cultures to be used without
  drastically changing the code base, enabling different reservoirs to be
  studied in a similar context.
\end{abstract}

% \begin{IEEEkeywords}
% neurons, reservoir-computing, cyborg
% \end{IEEEkeywords}

\section{Introduction}
Reservoir Computing is a novel approach to machine learning, using reservoirs
containing dynamic complex systems as computational elements. 
\cite{schrauwen_overview_2007}
Many different systems have been used as reservoirs, ranging from
virtual reservoirs such as random boolean networks and recurrent neural nets,
to real physical systems such as nano-carbon tubes.
Building on previous work we investigate using in-vitro neuron cultures as
reservoirs, describing the necessary infrastructure for performing such
experiments with live neurons. The primary concern of this paper will be the
architecture necessary for embodiment of blah blah blah
\section{Background}
\subsection{Reservoir computing}
A big disconnect between machine learning and how we perceive human reasoning is
how to solve and encode problems of temporal nature.
While feed forward structures which can only consider their current input can be
extended to reduce temporal problems into structural ones by adding delays
\cite{schrauwen_overview_2007}, a more natural approach is to use
recurrent structures that modify themselves in response to input.
A widely studied example of this approach is the recurrent neural network,
however the increase in expressiveness makes these structures very hard to train\cite{bertschinger_real-time_2004}.
Another example are random boolean networks \cite{gershenson_introduction_2004} studied by Kauffman as a
model for genetic regulatory networks.
Common for these two systems, and many more such as cellular
automata \cite{sipper_emergence_1999} and liquid state machines exhibit the
properties of \textit{complex systems} \cite{langton_computation_1990}.
Complex systems are systems that are, in a sense, magic.
In order to harness the power of these complex systems it is fruitless to
attempt to shape the topology and dynamics of the systems towards some specific
goal. Instead, the complex systems are used as \textit{reservoirs} which we will
interact with using a simple linearly separable input and output processing
layer which can be easily trained. \cite{schrauwen_overview_2007}
TO-DO: back linearly separable claims.
\subsection{Neurons}
Neurons are vastly complex entities, communicating through complex electric
and chemical signals. However, since we are more interested in the emergent
properties of neurons in the context of reservoir computing a superficial
description suffices.
We will only consider a generalized version of the neuron, but in our
experiments a plethora of different neurons are used, although they
all share the basic similarities described here.
The anatomy of a neuron is shown in \ref{fig:neuron_anatomy} and can roughly be
divided into the following parts:
\subsubsection{Soma}
The main body of the neuron. While we will view neurons as simple network nodes
it is important to note that the neuron is highly complex, it can blah blah
\subsubsection{Dendrites}
To sense its surroundings the neuron is equipped with dendrites. These
branching structures act as receivers, propagating electro-chemical stimuli to
the cell body. Their reach is only to the immediate vicinity of the cell, they
do not form longer connections.
\subsubsection{Axon}
The axon is a long tendril, extending over a meter in the case of the sciatic
nerve TO-DO maybe embed link? which transmits information as electrical pulses
to other neurons. An axon can branch off and reach multiple neurons, it is not a
one to one connection.
\subsection{The NTNU Cyborg Project}
\subsection{MEA2100}
To perform experiments a MEA2100 system has been purchased from multichannel systems.
The MEA2100 system is built to conduct experiments on in-vitro cell cultures, 
with the main focus being on neurons.
The principal components of the MEA2100 systems are:
\subsubsection{Micro electrode array}
\begin{figure}[h!]
    %\centering
    \includegraphics[width=\linewidth]{images/MEA.jpg}
    \caption{A generic MEA}
    \label{fig:generic_MEA}
\end{figure}
To experiment on the properties of cells or other electrically active subjects the
micro electrode array (MEA) is used. As the name implies the MEA is equipped with
an array of electrodes able to both measure the electrical properties of the 
experiment subjects, as well as applying outside stimuli, acting in a sense as
output and input for the subject. \ref{fig:generic_MEA} shows an empty MEA,
\ref{st_olav_MEA} shows an MEA from st.olavs with a live neuron culture.
\begin{figure}[h!]
    %\centering
    \includegraphics[width=\linewidth]{images/st-olavs-mea.jpg}
    \caption{A MEA with a live culture, photographed by Kai}
    \label{fig:st_olav_MEA}
\end{figure}
\subsubsection{Headstage}
The electrodes of the MEAs are measured and stimulated by the headstage which
contains the necessary high precision electronics needed for microvolt range readings.
\ref{fig:headstage} shows the same type of headstage used in this paper along
with an MEA.
\begin{figure}[h!]
    %\centering
    \includegraphics[width=\linewidth]{images/MEA2100-HS60.jpg}
    \caption{The headstage}
    \label{fig:headstage}
\end{figure}
\subsubsection{Interface board}
The interface board connects to up to two head-stages and is responsible for interfacing
with the data acquisition computer, as well as auxiliary equipment such as temperature
controls.
\begin{figure}[h!]
    %\centering
    \includegraphics[width=\linewidth]{images/MCS-IFB.jpg}
    \caption{The MCS interface board}
    \label{fig:neuron_anatomy}
\end{figure}
The interface board has two modes of operation.
In the first mode the interface board processes and filters data from up to two
headstages as shown in \ref{fig:IFB_regular} which can then be acquired on a normal
computer connected via USB.
In the second mode of operation a Texas instruments TMS320C6454 digital signal
processor is activated which can then be interfaced with using the secondary USB
port as shown in \ref{fig:IFB_BSP}
\begin{figure}[h!]
    %\centering
    \includegraphics[width=\linewidth]{images/regular_operation.png}
    \caption{Casual mode}
    \label{fig:IFB_regular}
\end{figure}
%% %% %% %% %% %% %% %% %% %% %% %% %% %% %% %% %% %% %% %% %% %% %% %% %% %%
%% %% %% %% %% %% %% %% %% %% %% %% %% %% %% %% %% %% %% %% %% %% %% %% %% %%
%% %% %% %% %% %% %% %% %% %% %% %% %% %% %% %% %% %% %% %% %% %% %% %% %% %%
%% %% %% %% %% %% %% %% %% %% %% %% %% %% %% %% %% %% %% %% %% %% %% %% %% %%
\begin{figure}[h!]
    %\centering
    \includegraphics[width=\linewidth]{images/dsp_operation.png}
    \caption{DSP active}
    \label{fig:IFB_DSP}
\end{figure}
%% %% %% %% %% %% %% %% %% %% %% %% %% %% %% %% %% %% %% %% %% %% %% %% %% %%
%% %% %% %% %% %% %% %% %% %% %% %% %% %% %% %% %% %% %% %% %% %% %% %% %% %%
%% %% %% %% %% %% %% %% %% %% %% %% %% %% %% %% %% %% %% %% %% %% %% %% %% %%
%% %% %% %% %% %% %% %% %% %% %% %% %% %% %% %% %% %% %% %% %% %% %% %% %% %%
\begin{figure*}[p]
    \centering
    \includegraphics[width=\textwidth]{images/"Neuron anatomy".png}
    \caption{a neuron}
    \label{fig:neuron_anatomy}
\end{figure*}
%%% Local Variables:
%%% mode: latex
%%% TeX-master: "../main"
%%% End:

\section{A Hybrid Neuro-Digital Approach}
A robot has been developed to this end, however it still lacks the biological
component required for it to be a real cybernetic organism.
\subsubsection{Concept}
The concept of the cyborg is inspired by similar efforts such as
\cite{li_application_2015} and \cite{warwick paper}.
Fig \[?\] shows the basic premise: An in vitro neuron culture will be used to
control a robot in a closed loop system, making the robot an actual cyborg.
\subsubsection{Platform}
The cyborg 
\begin{figure}[h!]
    %\centering
    \includegraphics[width=\linewidth]{images/cyborg_overview.png}
    \caption{The gist of it..}
    \label{fig:cyborg_idea}
\end{figure}
\subsubsection{Growing NIV}
\subsubsection{A First Test}

%%% Local Variables:
%%% mode: latex
%%% TeX-master: "../main"
%%% End:
\section{Results}
Describe initial tests running full closed loop system including visualizer bot.
\section{An RC Cyborg Platform}
Introductory the connection between neural networks and complex systems guided
the reservoir computational approach.
This fundamental similarity between seemingly unrelated systems has guided the
design process of the software system to allow for accommodating a wide variety
of reservoirs in contrast to previous approaches such as
\cite{li_application_2015}.
Rather than focusing simply on connecting in-vitro cultures to robots, the
architecture which is currently in development aims to provide a highly modular
system which allows for many different reservoirs on many different platforms.
By broadening the concept of a cyborg to a hybrid of machine and reservoir we
can study differences and similarities between different reservoirs, increasing
our understanding of the underlying properties of complex systems.
\subsection{Areas Of Concern}
The guiding principle for the RC cyborg architecture is to separate different
concerns and provide simple interfaces between them.
An important consequence of this choice is that it allows us to reuse much code
between a cyborg using a neural network as a reservoir and one using a random
boolean network.
To clarify it is important to introduce the concept of \textit{application
  specific} and \textit{reservoir specific} data processing.
The former is the data processing stage performed in context of reservoir
computing used to interpret the dynamics of the reservoir to perform a specific
task.
The latter describes data processing done outside the context of reservoir
computing, from simple noise reduction to more transformative filters, such as
spike detection for neural networks.
From this perspective the same neural network can act as two different
reservoirs, one delivering raw waveform data, the other delivering spike data.
The following sections describe the areas of concern, and where they fit in
context of reservoir computing.
\subsubsection{Data Acquisition and Interfacing}
Data acquisition (DAQ) is the task of configuring and interfacing with a specific
reservoir, essentially providing the wrapper for an actual reservoir which
represents the idealized shown in \ref{fig:RC}.
Finally, the data acquisition and interfacing module should expose an interface
via TCP/IP allowing configuring the reservoir, accessing reservoir output and
accepting requests to stimulate the reservoir.
\subsubsection{Data Processing}
This part of the architecture is responsible for the reservoir computing
processing of data which is shown in the idealized model of reservoir computing in
\ref{fig:RC}.
Implementations of the data processing stage must be able to filter input from
a reservoir received from the DAQ module into actions, as well as filtering
sensor data received from the agent control module to a format compatible with
the reservoir.
Note that two different reservoirs may provide the same interface, allowing the
same data processing component to utilize different reservoirs as long as the
format remains the same.
As with the DAQ module, data processing communicates via TCP/IP to both the DAQ
module and the agent control module.
\subsubsection{Agent Control}
This module implements the robot part of the RC-cyborg, which can range from
being a simple simulated agent in an idealized game to a fully fledged physical
robot interacting with the real world.
The agent control interfaces with the data processing module, receiving agent
specific output from the reservoir which it may use (in a physical robot the
reservoir does not have the final word as a safety measure).
It also transmits sensor data back to the data processing module, providing
feedback to the reservoir.
As with the DAQ module, some sensory processing may be done by agent control,
but only reservoir agnostic filtering.
\subsection{Implementation} 
The following sections describe the software currently implemented for
controlling the NTNU cyborg and where they fit into the general RC-cyborg
framework.
This represents the bulk of work performed in this paper.
\subsubsection{MEAME}
MEAME \cite{MEAME} implements data acquisition and interfacing for using neural cultures as a
reservoir.
It is written in C\# and interfaces to the MEA2100 system using an API provided
by multichannel systems allowing it to interface with the interface board
\ref{fig:MCS-IFB}.
MEAME also implements the optional DSP which is used to stimulate the neurons
and can be used to process the waveform data from the neural cultures in real
time, and even implement more advanced techniques for timing stimulation to
maximize impact.
By configuring the DSP the MEAME system may act as several reservoirs in context
of an RC-cyborg system, even though it uses the same underlying neural culture
for each mode.
Many configurations are possible for the DSP not only for processing outbound
data.
In \cite{kumar_autonomous_2016} machine learning is employed in order to select
optimal timings for stimulating a neural network to maximize response.
If this technique is implemented on the DSP the resulting system would be a
reservoir with different characteristics even though it uses the same underlying
neural network as its physical reservoir.
\subsubsection{SHODAN}
SHODAN \cite{SHODAN} is a framework for composing reservoir computing experiments written in
scala. In contrast to MEAME which is written specifically for interfacing with
MEA2100, SHODAN is intended to facilitate general purpose input and output
processing for reservoir computing.
The main focus of SHODAN is to perform task specific processing of reservoir
data, but it also comes with tools for extending reservoir specific processing
such as spike detection.
At the core SHODAN is a framework containing a library of tools used for
processing functional streams of data such as feed forward artificial neural
networks and methods for encoding and decoding data to facilitate implementing
different protocols for TCP/IP communication.
SHODAN also comes equipped with a built in simulator that currently consists of
a simple wall avoidance game which acts as an agent control module which
includes a real time visualizer that can be accessed in a browser.
%%% Local Variables:
%%% mode: latex
%%% TeX-master: "../main"
%%% End:
\section{Discussion}
\subsection{Reservoir Compatibility}
With the RC cyborg framework using different reservoirs in the same harness is
made possible, however this requires some modification to the reservoirs
themselves.
As an example we consider the difficulties involved with training the reservoir
computing system on neural networks.
These networks are constantly changing and growing, even small topological
changes may move an important cluster of neurons outside the range of an
electrode.
A possible approach is to use a random boolean network as a stand in reservoir for a
neural culture.
A random boolean network is a graph where each node is assigned a boolean function
of their inbound edges.
Since these nodes will have either the value 1 or 0 we can select a few of them
and let nodes with state 0 act as no activity, and nodes with the value 1 act as
a spiking neuron.
Similarly sensor stimuli can be used to switch input nodes on or off,
``perturbing'' the system.
Note that this is only one of many possible reservoir specific filters as
defined in the RC cyborg framework altering the functionality of the reservoir.
With this translation two very different underlying reservoirs can now be
interacted with using the same input and output, allowing a system to be trained
on different underlying reservoirs while sharing a majority of code.
Simply making the input and output between two reservoirs compatible will of
course not cause them to behave anything like each other.
The problem of utilizing random boolean networks as a viable stand-in for
neurons is very far from being solved, however compatibility is an important
first step.
\subsection{Adaptivity}
In the RC cyborg framework adaptivity and learning is not elaborated on, but
this aspect will become crucial when the implementations mature.
Reservoir computing does not add any constraints on how the task specific
filters should be constructed, thus typically the filters are created using
general purpose machine learning kits.
For reservoirs such as random boolean networks this is a sensible solution,
however for neural networks that evolve, changing their behavior with time the
RC cyborg must be capable of adapting along with the reservoir.
As with reservoir compatibility this is a complex task, but by considering the
learning system to be a part of the RC-cyborg we may lessen the compatibility
gap between different reservoirs.
\section{Conclusion}
A working solution for the architectural challenges for interfacing a robot with
a neural cultures has completed a proof of concept test, showing that the remaining
efforts for the cyborg project is harnessing the computations performed by the neural
network.
By proposing the RC cyborg concept as a broader definition of a cyborg the
foundation has been laid for exploring differences between neural networks and other
reservoirs used in the same contextual ``harness'' such that the differences between
reservoirs may be studied quantitatively as well as qualitatively.
With the theoretical and practical framework needed to interface with neural network
the cyborg project is now equipped to explore the fundamental aspects of neural
computation.

\section{Further Work}
For instance investigating growth rules for neurons in chaos or orderly
environments to investigate if they trend towards a critical phase.


\bibliography{mylib} 
\bibliographystyle{ieeetr}

\end{document}
