Introductory the connection between neural networks and complex systems guided
the reservoir computational approach.
This fundamental similarity between seemingly unrelated systems has guided the
design process of the software system to allow for accommodating a wide variety
of reservoirs in contrast to previous approaches such as
\cite{li_application_2015}.
Rather than focusing simply on connecting in-vitro cultures to robots, the
architecture which is currently in development aims to provide a highly modular
system which allows for many different reservoirs on many different platforms.
By broadening the concept of a cyborg to a hybrid of machine and reservoir we
can study differences and similarities between different reservoirs, increasing
our understanding of the underlying properties of complex systems.
\subsection{Areas Of Concern}
The guiding principle for the RC cyborg architecture is to separate different
concerns and provide simple interfaces between them.
An important consequence of this choice is that it allows us to reuse much code
between a cyborg using a neural network as a reservoir and one using a random
boolean network.
To clarify it is important to introduce the concept of \textit{application
  specific} and \textit{reservoir specific} data processing.
The former is the data processing stage performed in context of reservoir
computing used to interpret the dynamics of the reservoir to perform a specific
task.
The latter describes data processing done outside the context of reservoir
computing, from simple noise reduction to more transformative filters, such as
spike detection for neural networks.
From this perspective the same neural network can act as two different
reservoirs, one delivering raw waveform data, the other delivering spike data.
The following sections describe the areas of concern, and where they fit in
context of reservoir computing.
\subsubsection{Data Acquisition and Interfacing}
Data acquisition (DAQ) is the task of configuring and interfacing with a specific
reservoir, essentially providing the wrapper for an actual reservoir which
represents the idealized shown in \ref{that RC image}.
The data acquisition may perform reservoir specific data processing as described
in the previous section, 
% ???
Finally, the data acquisition and interfacing module should expose an interface
via TCP/IP allowing configuring the reservoir, accessing reservoir output and
accepting requests to stimulate the reservoir.
%
% Data acquisition entails configuring the MEA2100, collecting 
% data from the MEA2100 and setting up stimuli. This means the data acquisition
% and interfacing software must at least be able to deliver a raw, that is
% unprocessed data-stream, as well as being able to accept requests for stimuli
% over TCP/IP.
% In practice the data acquisition software may also handle processing of acquired
% data for performance reasons, but it is not a required task.
\subsubsection{Data Processing}
This part of the architecture is responsible for the reservoir computing
processing of data which is shown in the idealized model of reservoir computing in
\ref{that RC image}.
Implementations of the data processing stage must be able to filter input from
a reservoir received from the DAQ module into actions, as well as filtering
sensor data received from the agent control module to a format compatible with
the reservoir.
Note that two different reservoirs may provide the same interface, allowing the
same data processing component to utilize different reservoirs as long as the
format remains the same.
As with the DAQ module, data processing communicates via TCP/IP to both the DAQ
module and the agent control module.
\subsubsection{Agent Control}
This module implements the robot part of the RC-cyborg, which can range from
being a simple simulated agent in an idealized game to a fully fledged physical
robot interacting with the real world.
The agent control interfaces with the data processing module, receiving agent
specific output from the reservoir which it may use (in a physical robot the
reservoir does not have the final word as a safety measure).
It also transmits sensor data back to the data processing module, providing
feedback to the reservoir.
As with the DAQ module, some sensory processing may be done by agent control,
but only reservoir agnostic filtering.
\subsection{Implementation} 
The following sections describe the software currently implemented for
controlling the NTNU cyborg and where they fit into the general RC-cyborg
framework.
This represents the bulk of work performed in this paper.
\subsubsection{MEAME}
MEAME implements data acquisition and interfacing for using neural cultures as a
reservoir.
It is written in C\# and interfaces to the MEA2100 system using an API provided
by multichannel systems allowing it to interface with the interface board
\ref{fig:IFB}.
MEAME also implements the optional DSP which is used to stimulate the neurons
and can be used to process the waveform data from the neural cultures in real
time, and even implement more advanced techniques for timing stimulation to
maximize impact.
By configuring the DSP the MEAME system may act as several reservoirs in context
of an RC-cyborg system, even though it uses the same underlying neural culture
for each mode.
Many configurations are possible for the DSP not only for processing outbound
data.
In \cite{kumar_autonomous_2016} machine learning is employed in order to select
optimal timings for stimulating a neural network to maximize response.
If this technique is implemented on the DSP the resulting system would be a
reservoir with different characteristics even though it uses the same underlying
neural network as its physical reservoir.
\subsubsection{SHODAN}
SHODAN is a framework for composing reservoir computing experiments written in
scala. In contrast to MEAME which is written specifically for interfacing with
MEA2100, SHODAN is intended to facilitate general purpose input and output
processing for reservoir computing.
The main focus of SHODAN is to perform task specific processing of reservoir
data, but it also comes with tools for extending reservoir specific processing
such as spike detection.
At the core SHODAN is a framework containing a library of tools used for
processing functional streams of data such as feed forward artificial neural
networks and methods for encoding and decoding data to facilitate implementing
different protocols for TCP/IP communication.
SHODAN also comes equipped with a built in simulator that currently consists of
a simple wall avoidance game which acts as an agent control module which
includes a real time visualizer that can be accessed in a browser.
%%% Local Variables:
%%% mode: latex
%%% TeX-master: "../main"
%%% End: