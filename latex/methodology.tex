TO-DO: differentiate RC-agnostic post processing and spike detection.
Should probably be done in background

The architecture described in this paper is a refinement of the neuro-robot
architecture used in \cite{li_application_2015}.
In \cite{li_application_2015} a system for working specifically with MEAs
containing dissociated neurons is described with a TCP/IP connection to a slave
PC controlling a robot being the main selling point. Our model is a
generalization of \cite{li_application_2015} designed for reservoir computing,
where the neuron-culture is just one of many possible reservoirs.
It should be noted that our architecture does not require experiments to be
posed in the context of reservoir computing. (not too happy with the words here,
make them betterer. Also something about the possibility of multiple implementations)
\subsection{Areas Of Concern}
The architecture has been defined by three areas of concern. This modularization
facilitates reuse of code ??? some words ???
In this section it is important to note that we differentiate between generic
data-processing and task specific processing. The former describes ~~getting
spike trains while the latter describes reservoir computing specific processing~~
~~This is good because it separates RC from other stuff~~
\subsubsection{Data Acquisition and Interfacing}
Data acquisition entails configuring the MEA2100, collecting 
data from the MEA2100 and setting up stimuli. This means the data acquisition
and interfacing software must at least be able to deliver a raw, that is
unprocessed data-stream, as well as being able to accept requests for stimuli
over TCP/IP.
In practice the data acquisition software may also handle processing of acquired
data for performance reasons, but it is not a required task.
\subsubsection{Data Processing}
This part of the architecture is responsible for processing data in a task
specific, i.e non-generic way. The data may be raw waveform data or spike data
from the MEA that needs to be processed into commands for an agent, or it may be
sensor data from a simulated agent that must be processed into stimulus requests
for the MEA. As with the data acquisition and interfacing module, the data
processing module communicates over TCP/IP.
\subsubsection{Agent Control}
This module implements the actual embodiment of the neuron-culture. The agent
control reads the processed data from the data processing modules and issues
commands to an agent. It also transmits sensor data back to the data processing
module, providing feedback to the neuron culture.
The type of the agent is not specified, it can be a fully fledged walkin'
talkin' cyborg, or it can be a simple simulated agent.
\subsection{Implementation} 
Currently the following software systems have been implemented:
\subsubsection{MEAME}
MEAME implements the data acquisition and interfacing requirement of the closed
loop system. It is written in C\# and interfaces with an API provided by
multichannel systems which allows it to interface with the MCS interface board.
MEAME also implements the optional DSP which is used to stimulate the neurons
thus fulfilling both the requirements.
\subsubsection{SHODAN}
SHODAN is a framework for composing reservoir computing experiments written in
scala. In contrast to MEAME which is written specifically for interfacing with
MEA2100, SHODAN is intended to facilitate general purpose input and output
processing for reservoir computing.
SHODAN currently hosts a number of tools for composing and reusing
data-processing pipelines such as parametrizable feed forward neural nets,
serialization and deserialization methods for TCP streams for communicating with
the data-acquisition module and the agent control module.
Additionally SHODAN hosts a simple simulator intended to stand in for a more
full-fledged agent control module. This simple simulator simulates a simple
agent with four eyes which reacts to proximity to a wall, and can visualized in
a browser in real-time to get that sweet grant \$\$\$
\subsection{Interfacing With the MEA2100 redux}

%%% Local Variables:
%%% mode: latex
%%% TeX-master: "../main"
%%% End: