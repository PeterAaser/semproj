Since the 50s, the Von-Neumann computer architecture has been ubiquitous in the
field of computing. This architecture owes its success to its relative
simplicity, allowing quick iterations in tact with doubling transistor counts in
accord with moores law.
However several issues with the Von-Neumann architecture have become more and more
pressing in recent times. Issues such as the Von-Neumann memory bottleneck are
becoming increasingly difficult to hide, and its inherently serial nature of computation.
Even in the case that moores law will continue, we still face issues with lack
of scalability in modern processors.
With billions of transistors a top down
design process is becoming increasingly difficult and expensive, and a single
faulty transistor can cause an entire chip to be useless.
These weaknesses necessitates going beyond the Von-Neumann architecture,
exploring unconventional computing paradigms.
Taking inspiration from nature we
look at cellular computing consisting of computational cells which by themselves
are unremarkable.
In these systems we observe properties that cannot be
traced back to a single cell, they only appear when multiple cells are
interacting with each other.
In nature these systems form without any form of supervision, a property known
as \textit{self organization}, with an example being 
DNA which ``describes how to build the system, not what the system will look
like'' \cite{tufte_evo_2009}.
These \textit{emergent properties} arise in systems built from the bottom up by
%
a set of growth rules rather than a designers intent, a  property, which allows a flexibility and robustness that modern
processors lack, a prime example being the human brain.
%
The human brain is a vastly parallel computer, eclipsing modern processors in
terms of computational capacity, robustness, energy efficiency and complexity.
The three first properties make them very appealing subjects to study for
unconventional computing, however due to the immense complexity the brain is
still enigmatic.
In this paper we explore using neuron cultures to create an organic mechanical
hybrid organism, a so called \textit{cyborg}.

%%% Local Variables:
%%% mode: latex
%%% TeX-master: "../main"
%%% End: