Since the 50s, the Von-Neumann computer architecture has been ubiquitous in the
field of computing. This architecture owes its success to its relative
simplicity, allowing quick iterations in tact with doubling transistor counts in
accord with moores law \cite{moore}.
However several issues with the Von Neumann architecture have become more and more
pressing in recent times.
Issues such as the Von-Neumann memory bottleneck are becoming increasingly
difficult to hide, and its inherently serial nature of computation makes
increasing parallelism difficult and error prone.
HiPEAC 2015\cite{HIPEAC} identifies four key challenges for the future of
computer architecture design.
\begin{enumerate}
\item dependability by design
\item managing system complexity
\item energy efficiency
\item entanglement between physical and virtual world
\end{enumerate}
Even with the assumption that moores law will hold in the future these issues
threaten to drastically reduce the scalability of modern processor designs.
These weaknesses necessitate going beyond the Von-Neumann architecture,
exploring unconventional computing paradigms.
In nature a completely different approach is taken which effectively deal with
all these issues:
The human brain vastly outperform modern computers on the three first issues, having no
single point of failure, being self organized and operating at an estimate of 20 Watts.
Not only does the neurons making up the brain easily surpass computers, the same
neurons provide a sense of touch, sight and smell handily beating current
efforts in the fourth issue.
Taking inspiration from nature we look at the paradigm of \textit{cellular computing}
\cite{sipper_emergence_1999} where cells which by themselves are unremarkable
are combined to perform highly efficient parallel computations.
These cellular systems exhibit behavior properties that cannot be
traced back to a single cell, they only appear when multiple cells are
interacting with each other, known as \textit{emergent behavior}.
In nature these systems form without any form of supervision, a property known
as \textit{self organization}, with an example being
DNA which ``describes how to build the system, not what the system will look
like'' \cite{tufte_evo_2009}.
Understanding how these cellular systems that arise seemingly by
themselves can be so successful and how we can harness this power may hold the
key to solving problems faced in modern computer architectures, and may lead to
advancements in neuroscience as well.
In this paper we explore using neuron cultures to create an organic mechanical
hybrid organism, a so called \textit{cyborg} as part of the NTNU cyborg project.
%%% Local Variables:
%%% mode: latex
%%% TeX-master: "../main"
%%% End: