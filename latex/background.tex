\subsection{The NTNU Cyborg Project}
The NTNU cyborg project is a collaboration between several departments at NTNU
including the department of biotechnology, computer and information science,
engineering cybernetics, neuroscience and more. \cite{fossen_ntnu_???}
The stated goal for the cyborg project is ``to enable communication between living
nerve tissue and a robot. The social and interactive cyborg will walk around the campus
raising awareness for biotechnology and ICT, bringing NTNU in the forefront of research
and creating a platform for interdisciplinary collaborations and teaching.''
Currently the department of neuroscience is growing neuron cultures which are to
be used as the biological part of the robot.
These neuron cultures are not part of a brain, they are fully dissociated, grown
in special chambers in-vitro.
The robot part of the cyborg has been developed and implemented by the
department of engineering cybernetics, and is currently operational, however it
has not yet been integrated with the in-vitro neuron cultures.
The challenge faced by the cyborg project is the infrastructure for interfacing the
neuron cultures and the robot, essentially creating a brain computer interface.
\subsection{Reservoir computing}
\textit{
  In this paper references are made to computations done by both artificial and real
  neurons.
  To make the distinction between these cases clear all computation done by
  computer simulated approximations of neurons will be prefixed as artificial.
}\\
The emergent properties that makes cellular computing a such a successful
paradigm in nature are also the reason that exploiting such systems has proven
difficult.
Many such systems have been explored, such as recurrent artificial neural
networks \cite{bertschinger_real-time_2004}, random boolean networks
\cite{gershenson_introduction_2004} and cellular automata.
In \cite{langton_computation_1990} Langton explores the requirements for systems
to support computation. He argues that in order for a system to support
spontaneous computation it must lie blah blah blah

Harnessing this emergent computational capacity has however turned out to be
very hard.
The blah blah fitness landscape.
Instead of attempting to directly shape such a system, a much more tractable
approach is to treat it as a \textit{reservoir} \cite{schrauwen_overview_2007} 
which ``acts as a complex nonlinear dynamic filter that transforms the
input signals using a high-dimensional temporal map, not unlike the operation
of an explicit, temporal kernel function.''
With this approach we shift our focus towards

\ref{fig:RC} shows a typical RC (reservoir computing) system with three inputs
and two outputs. The inputs are processed in a simple feed-forward neural
network before perturbing the reservoir in some way.
Similarly the state of the reservoir is being processed by an output layer
before leaving the RC system.
In the figure the input and output processing is done by feed forward neural
networks, but we note that this is only one of many possible filters.
Inputs 1, 2 and 3 are snapshots of the current state of the problem we attempt
to solve with reservoir computing.
Since our filters have no state, at least not beyond some time horizon we see
that the history of the system must in some way be encoded in the reservoir in
order for the RC system to solve problems in scope wider than the limited amount
of state that may be contained in the filters.
\begin{figure*}[h]
  % \begin{tikzpicture}[scale=3]
%   \clip (-2,-0.3) rectangle (2,0.9);
%   \draw[step=.5cm,gray,very thin] (-1.4,-1.4) grid (1.4,1.4);
%   \filldraw[fill=green!20,draw=green!50!black] (0,0) -- (3mm,0mm)
%     arc [start angle=0, end angle=30, radius=3mm] -- cycle;
% 
%   \draw[->] (-1.5,0) -- (1.5,0) coordinate (x axis);
%   \draw[->] (0,-1.5) -- (0,1.5) coordinate (y axis);
%   \draw (0,0) circle [radius=1cm];
%   \draw[very thick,red]
%     (30:1cm) -- node[left=1pt,fill=white] {$\sin \alpha$} (30:1cm |- x axis);
% 
%   \draw[very thick,blue]
%     (30:1cm |- x axis) -- node[below=2pt,fill=white] {$\cos \alpha$} (0,0);
% 
%   \path [name path=upward line] (1,0) -- (1,1);
%   \path [name path=sloped line] (0,0) -- (30:1.5cm);
%   \draw [name intersections={of=upward line and sloped line, by=t}]
%     [very thick,orange] (1,0) -- node [right=1pt,fill=white]
%     {$\displaystyle \tan \alpha \color{black}=
%       \frac{{\color{red}\sin \alpha}}{\color{blue}\cos \alpha}$} (t);
% 
%   \draw (0,0) -- (t);
%   \foreach \x/\xtext in {-1, -0.5/-\frac{1}{2}, 1}
%   \draw (\x cm,1pt) -- (\x cm,-1pt) node[anchor=north,fill=white] {$\xtext$};
%   \foreach \y/\ytext in {-1, -0.5/-\frac{1}{2}, 0.5/\frac{1}{2}, 1}
%   \draw (1pt,\y cm) -- (-1pt,\y cm) node[anchor=east,fill=white] {$\ytext$};
% \end{tikzpicture}
\pgfdeclarelayer{background}
\pgfdeclarelayer{foreground}
\pgfsetlayers{background,main,foreground}

% Define block styles used later
\tikzstyle{sensor}=[draw, fill=blue!20, text width=5em, 
    text centered, minimum height=2.5em,drop shadow]

\tikzstyle{ann} = [above, text width=5em, text centered]

\tikzstyle{wa} = [sensor, text width=10em, fill=red!20, 
    minimum height=6em, rounded corners, drop shadow]

\tikzstyle{sc} = [sensor, text width=13em, fill=red!20, 
    minimum height=10em, rounded corners, drop shadow]

% Define distances for bordering
\def\blockdist{2.3}
\def\edgedist{2.5}

\begin{figure}[p]
\begin{tikzpicture}
    \node (wa) [wa]  {System Combination};
    \path (wa.west)+(-3.2,1.5) node (asr1) [sensor] {$ASR_1$};
    \path (wa.west)+(-3.2,0.5) node (asr2) [sensor] {$ASR_2$};
    \path (wa.west)+(-3.2,-1.0) node (dots) [ann] {$\vdots$}; 
    \path (wa.west)+(-3.2,-2.0) node (asr3)[sensor] {$ASR_N$};    
   
    \path (wa.east)+(\blockdist,0) node (vote) [sensor] {$\theta_0,\theta_1,...,\theta_M$\\Estimated Parameters};

    \path [draw, ->] (asr1.east) -- node [above] {} 
        (wa.160) ;
    \path [draw, ->] (asr2.east) -- node [above] {} 
        (wa.180);
    \path [draw, ->] (asr3.east) -- node [above] {} 
        (wa.200);
    \path [draw, ->] (wa.east) -- node [above] {} 
        (vote.west);

               
    \path (wa.south) +(0,-\blockdist) node (asrs) {System Combination - Training};
  
    \begin{pgfonlayer}{background}
        \path (asr1.west |- asr1.north)+(-0.5,0.3) node (a) {};
        %\path (wa.south -| wa.east)+(+0.5,-0.3) node (b) {};
        \path (vote.east |- asrs.east)+(+0.5,-0.5) node (c) {};
          
        \path[fill=yellow!20,rounded corners, draw=black!50, dashed]
            (a) rectangle (c);           
        \path (asr1.north west)+(-0.2,0.2) node (a) {};
            
    \end{pgfonlayer}
    
    % Validation Layer is the same except that there are a set of nodes and links which are added
   

    % \path (wa.south)+(-2.0,-7.5) node (syscomb) [sc] {\textbf{System Combination \\Algorithm}\\Estimated Parameters\\from training};
    % \path (syscomb.west)+(-2.2,1.5) node (asrt1) [sensor] {$ASR_1$};
    % \path (syscomb.west)+(-2.2,0.5) node (asrt2)[sensor] {$ASR_2$};
    % \path (syscomb.west)+(-2.2,-1.0) node (dots)[ann] {$\vdots$}; 
    % \path (syscomb.west)+(-2.2,-2.0) node (asrt3)[sensor] {$ASR_N$};    

    % \path [draw, ->] (asrt1.east) -- node [above] {} 
    %     (syscomb.160) ;
    % \path [draw, ->] (asrt2.east) -- node [above] {} 
    %     (syscomb.180);
    % \path [draw, ->] (asrt3.east) -- node [above] {} 
    %     (syscomb.200);

    %            
    % \path (wa.south) +(0,-\blockdist) node (sct) {System Combination - Training};
 

    % \path (syscomb.east)+(1.0,0.0) node (bwtn) {};

    % % Note how the single nodes are repeated using for loop
    % \foreach \x in {0,1,...,4} 
    % { 
    %     \draw (bwtn.east)+(\x,0) node (asr\x-2)[]{}; 
    %     \fill (bwtn.east)+(\x,0) circle (0.1cm); 
    % }
   
    % \path [draw, ->] (syscomb.east) -- node [above] {} 
    %     (bwtn.east);
	  % \path [draw, ->] (asr0-2) -- node [above] {@} 
    %     (asr1-2);
    % \path [draw, -] (asr1-2) -- node [above] {b} 
    %     (asr2-2);
    % \path [draw, -] (asr2-2) -- node [above] {z} 
    %     (asr3-2);
    % \path [draw, -] (asr3-2) -- node [above] {} 
    %     (asr4-2);

    % \path [draw, ->] (asr0-2) edge[bend  right]  node [below] {@} 
    %     (asr1-2);
    % \path [draw, ->] (asr1-2) edge[bend  right]  node [below] {b} 
    %     (asr2-2);
    % \path [draw, ->] (asr2-2) edge[bend  right]  node [below] {c} 
    %     (asr3-2);
    % \path [draw, ->] (asr4-2) node[]{} (asr4-2)+(1.0,0);

    % \begin{scope}[looseness=1.6]
    %     \path [draw, ->] (asr0-2) edge[bend  right=90]  node [below] {a} 
    %         (asr1-2);
    %     \path [draw, ->] (asr1-2) edge[bend  right=90]  node [below] {b} 
    %         (asr2-2);
    %     \path [draw, ->] (asr2-2) edge[bend  right=90]  node [below] {c} 
    %         (asr3-2);
    % \end{scope}
    % \path (asr3-2.east)+(1.5,0.0) node (bw)[sensor] {Best Word Sequence\\$\arg\max$};    

    % \path [draw, -] (asr1-2.east) node [below] {} 
    %     (bw.west);
    %       
    % \begin{pgfonlayer}{background}
    %     \path (asrt1.west)+(-0.5,1.0) node (g) {};
    %     \path (bw.east |- syscomb.south)+(0.5,-1.5) node (h) {};
    %      
    %     \path[fill=yellow!20,rounded corners, draw=black!50, dashed]
    %         (g) rectangle (h);

    %     \path [draw, ->] (vote.south) edge[bend  left=90]  node [below] {Used in validation} 
    %         (syscomb.30);            

    % \end{pgfonlayer}
    % 
    % \path (asr1-2.south) +(-\blockdist,-\blockdist) 
    %     node (asrs) {System Combination - Validation};

  \end{tikzpicture}
\end{figure}
  \caption{A reservoir comput-thingy}
  \label{fig:RC}
\end{figure*}
\subsection{Neuron Computing}
Neurons are vastly complex entities, communicating through complex electric
and chemical signals.
The human brain consist of 100 billion neurons, however useful computation can
occur with far fewer neurons, as evident from the hierarchical structure of the
human brain.

However, since we are more interested in the emergent properties of neurons in
the context of reservoir computing a superficial description suffices.
We will only consider a generalized version of the neuron, but in our
experiments a plethora of different neurons are used, although they
all share the basic similarities described here.
The anatomy of a neuron is shown in \ref{fig:neuron_anatomy} and can roughly be
divided into the following parts:
\subsubsection{Soma}
The main body of the neuron. While we will view neurons as simple network nodes
it is important to note that the neuron is highly complex, it can blah blah
\subsubsection{Dendrites}
To sense its surroundings the neuron is equipped with dendrites. These
branching structures act as receivers, propagating electro-chemical stimuli to
the cell body. Their reach is only to the immediate vicinity of the cell, they
do not form longer connections.
\subsubsection{Axon}
The axon is a long tendril, extending over a meter in the case of the sciatic
nerve, which transmits information as electrical pulses to other neurons. An
axon can branch off and reach multiple neurons, it is not a one to one
connection.
\subsection{Adaptivity}

\subsection{MEA2100}
To perform experiments a MEA2100 system has been purchased from multichannel systems.
The MEA2100 system is built to conduct experiments on in-vitro cell cultures, 
with the main focus being on neurons.
The principal components of the MEA2100 systems are:
\subsubsection{Micro electrode array}
\begin{figure}[h!]
    %\centering
    \includegraphics[width=\linewidth]{images/MEA.jpg}
    \caption{A generic MEA}
    \label{fig:generic_MEA}
\end{figure}
To experiment on the properties of cells or other electrically active subjects the
micro electrode array (MEA) is used. As the name implies the MEA is equipped with
an array of electrodes able to both measure the electrical properties of the 
experiment subjects, as well as applying outside stimuli, acting in a sense as
output and input for the subject. \ref{fig:generic_MEA} shows an empty MEA,
\ref{st_olav_MEA} shows an MEA from st.olavs with a live neuron culture.
\begin{figure}[h!]
    %\centering
    \includegraphics[width=\linewidth]{images/st-olavs-mea.jpg}
    \caption{A MEA with a live culture, photographed by Kai}
    \label{fig:st_olav_MEA}
\end{figure}
\subsubsection{Headstage}
The electrodes of the MEAs are measured and stimulated by the headstage which
contains the necessary high precision electronics needed for microvolt range readings.
\ref{fig:headstage} shows the same type of headstage used in this paper along
with an MEA.
\begin{figure}[h!]
    %\centering
    \includegraphics[width=\linewidth]{images/MEA2100-HS60.jpg}
    \caption{The headstage}
    \label{fig:headstage}
\end{figure}
\subsubsection{Interface board}
The interface board connects to up to two head-stages and is responsible for interfacing
with the data acquisition computer, as well as auxiliary equipment such as temperature
controls.
\begin{figure}[h!]
    %\centering
    \includegraphics[width=\linewidth]{images/MCS-IFB.jpg}
    \caption{The MCS interface board}
    \label{fig:neuron_anatomy}
\end{figure}
The interface board has two modes of operation.
In the first mode the interface board processes and filters data from up to two
headstages as shown in \ref{fig:IFB_regular} which can then be acquired on a normal
computer connected via USB.
In the second mode of operation a Texas instruments TMS320C6454 digital signal
processor is activated which can then be interfaced with using the secondary USB
port as shown in \ref{fig:IFB_BSP}
\begin{figure}[h!]
    %\centering
    \includegraphics[width=\linewidth]{images/regular_operation.png}
    \caption{Casual mode}
    \label{fig:IFB_regular}
\end{figure}
%% %% %% %% %% %% %% %% %% %% %% %% %% %% %% %% %% %% %% %% %% %% %% %% %% %%
%% %% %% %% %% %% %% %% %% %% %% %% %% %% %% %% %% %% %% %% %% %% %% %% %% %%
%% %% %% %% %% %% %% %% %% %% %% %% %% %% %% %% %% %% %% %% %% %% %% %% %% %%
%% %% %% %% %% %% %% %% %% %% %% %% %% %% %% %% %% %% %% %% %% %% %% %% %% %%
\begin{figure}[h!]
    %\centering
    \includegraphics[width=\linewidth]{images/dsp_operation.png}
    \caption{DSP active}
    \label{fig:IFB_DSP}
\end{figure}
%% %% %% %% %% %% %% %% %% %% %% %% %% %% %% %% %% %% %% %% %% %% %% %% %% %%
%% %% %% %% %% %% %% %% %% %% %% %% %% %% %% %% %% %% %% %% %% %% %% %% %% %%
%% %% %% %% %% %% %% %% %% %% %% %% %% %% %% %% %% %% %% %% %% %% %% %% %% %%
%% %% %% %% %% %% %% %% %% %% %% %% %% %% %% %% %% %% %% %% %% %% %% %% %% %%
\begin{figure*}[p]
    \centering
    \includegraphics[width=\textwidth]{images/"Neuron anatomy".png}
    \caption{a neuron}
    \label{fig:neuron_anatomy}
\end{figure*}
%%% Local Variables:
%%% mode: latex
%%% TeX-master: "../main"
%%% End: