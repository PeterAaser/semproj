% \begin{tikzpicture}[scale=3]
%   \clip (-2,-0.3) rectangle (2,0.9);
%   \draw[step=.5cm,gray,very thin] (-1.4,-1.4) grid (1.4,1.4);
%   \filldraw[fill=green!20,draw=green!50!black] (0,0) -- (3mm,0mm)
%     arc [start angle=0, end angle=30, radius=3mm] -- cycle;
% 
%   \draw[->] (-1.5,0) -- (1.5,0) coordinate (x axis);
%   \draw[->] (0,-1.5) -- (0,1.5) coordinate (y axis);
%   \draw (0,0) circle [radius=1cm];
%   \draw[very thick,red]
%     (30:1cm) -- node[left=1pt,fill=white] {$\sin \alpha$} (30:1cm |- x axis);
% 
%   \draw[very thick,blue]
%     (30:1cm |- x axis) -- node[below=2pt,fill=white] {$\cos \alpha$} (0,0);
% 
%   \path [name path=upward line] (1,0) -- (1,1);
%   \path [name path=sloped line] (0,0) -- (30:1.5cm);
%   \draw [name intersections={of=upward line and sloped line, by=t}]
%     [very thick,orange] (1,0) -- node [right=1pt,fill=white]
%     {$\displaystyle \tan \alpha \color{black}=
%       \frac{{\color{red}\sin \alpha}}{\color{blue}\cos \alpha}$} (t);
% 
%   \draw (0,0) -- (t);
%   \foreach \x/\xtext in {-1, -0.5/-\frac{1}{2}, 1}
%   \draw (\x cm,1pt) -- (\x cm,-1pt) node[anchor=north,fill=white] {$\xtext$};
%   \foreach \y/\ytext in {-1, -0.5/-\frac{1}{2}, 0.5/\frac{1}{2}, 1}
%   \draw (1pt,\y cm) -- (-1pt,\y cm) node[anchor=east,fill=white] {$\ytext$};
% \end{tikzpicture}
\pgfdeclarelayer{background}
\pgfdeclarelayer{foreground}
\pgfsetlayers{background,main,foreground}

% Define block styles used later
\tikzstyle{sensor}=[draw, fill=blue!20, text width=5em, 
    text centered, minimum height=2.5em,drop shadow]

\tikzstyle{ann} = [above, text width=5em, text centered]

\tikzstyle{wa} = [sensor, text width=10em, fill=red!20, 
    minimum height=6em, rounded corners, drop shadow]

\tikzstyle{sc} = [sensor, text width=13em, fill=red!20, 
    minimum height=10em, rounded corners, drop shadow]

% Define distances for bordering
\def\blockdist{2.3}
\def\edgedist{2.5}

\begin{figure}[p]
\begin{tikzpicture}
    \node (wa) [wa]  {System Combination};
    \path (wa.west)+(-3.2,1.5) node (asr1) [sensor] {$ASR_1$};
    \path (wa.west)+(-3.2,0.5) node (asr2) [sensor] {$ASR_2$};
    \path (wa.west)+(-3.2,-1.0) node (dots) [ann] {$\vdots$}; 
    \path (wa.west)+(-3.2,-2.0) node (asr3)[sensor] {$ASR_N$};    
   
    \path (wa.east)+(\blockdist,0) node (vote) [sensor] {$\theta_0,\theta_1,...,\theta_M$\\Estimated Parameters};

    \path [draw, ->] (asr1.east) -- node [above] {} 
        (wa.160) ;
    \path [draw, ->] (asr2.east) -- node [above] {} 
        (wa.180);
    \path [draw, ->] (asr3.east) -- node [above] {} 
        (wa.200);
    \path [draw, ->] (wa.east) -- node [above] {} 
        (vote.west);

               
    \path (wa.south) +(0,-\blockdist) node (asrs) {System Combination - Training};
  
    \begin{pgfonlayer}{background}
        \path (asr1.west |- asr1.north)+(-0.5,0.3) node (a) {};
        %\path (wa.south -| wa.east)+(+0.5,-0.3) node (b) {};
        \path (vote.east |- asrs.east)+(+0.5,-0.5) node (c) {};
          
        \path[fill=yellow!20,rounded corners, draw=black!50, dashed]
            (a) rectangle (c);           
        \path (asr1.north west)+(-0.2,0.2) node (a) {};
            
    \end{pgfonlayer}
    
    % Validation Layer is the same except that there are a set of nodes and links which are added
   

    % \path (wa.south)+(-2.0,-7.5) node (syscomb) [sc] {\textbf{System Combination \\Algorithm}\\Estimated Parameters\\from training};
    % \path (syscomb.west)+(-2.2,1.5) node (asrt1) [sensor] {$ASR_1$};
    % \path (syscomb.west)+(-2.2,0.5) node (asrt2)[sensor] {$ASR_2$};
    % \path (syscomb.west)+(-2.2,-1.0) node (dots)[ann] {$\vdots$}; 
    % \path (syscomb.west)+(-2.2,-2.0) node (asrt3)[sensor] {$ASR_N$};    

    % \path [draw, ->] (asrt1.east) -- node [above] {} 
    %     (syscomb.160) ;
    % \path [draw, ->] (asrt2.east) -- node [above] {} 
    %     (syscomb.180);
    % \path [draw, ->] (asrt3.east) -- node [above] {} 
    %     (syscomb.200);

    %            
    % \path (wa.south) +(0,-\blockdist) node (sct) {System Combination - Training};
 

    % \path (syscomb.east)+(1.0,0.0) node (bwtn) {};

    % % Note how the single nodes are repeated using for loop
    % \foreach \x in {0,1,...,4} 
    % { 
    %     \draw (bwtn.east)+(\x,0) node (asr\x-2)[]{}; 
    %     \fill (bwtn.east)+(\x,0) circle (0.1cm); 
    % }
   
    % \path [draw, ->] (syscomb.east) -- node [above] {} 
    %     (bwtn.east);
	  % \path [draw, ->] (asr0-2) -- node [above] {@} 
    %     (asr1-2);
    % \path [draw, -] (asr1-2) -- node [above] {b} 
    %     (asr2-2);
    % \path [draw, -] (asr2-2) -- node [above] {z} 
    %     (asr3-2);
    % \path [draw, -] (asr3-2) -- node [above] {} 
    %     (asr4-2);

    % \path [draw, ->] (asr0-2) edge[bend  right]  node [below] {@} 
    %     (asr1-2);
    % \path [draw, ->] (asr1-2) edge[bend  right]  node [below] {b} 
    %     (asr2-2);
    % \path [draw, ->] (asr2-2) edge[bend  right]  node [below] {c} 
    %     (asr3-2);
    % \path [draw, ->] (asr4-2) node[]{} (asr4-2)+(1.0,0);

    % \begin{scope}[looseness=1.6]
    %     \path [draw, ->] (asr0-2) edge[bend  right=90]  node [below] {a} 
    %         (asr1-2);
    %     \path [draw, ->] (asr1-2) edge[bend  right=90]  node [below] {b} 
    %         (asr2-2);
    %     \path [draw, ->] (asr2-2) edge[bend  right=90]  node [below] {c} 
    %         (asr3-2);
    % \end{scope}
    % \path (asr3-2.east)+(1.5,0.0) node (bw)[sensor] {Best Word Sequence\\$\arg\max$};    

    % \path [draw, -] (asr1-2.east) node [below] {} 
    %     (bw.west);
    %       
    % \begin{pgfonlayer}{background}
    %     \path (asrt1.west)+(-0.5,1.0) node (g) {};
    %     \path (bw.east |- syscomb.south)+(0.5,-1.5) node (h) {};
    %      
    %     \path[fill=yellow!20,rounded corners, draw=black!50, dashed]
    %         (g) rectangle (h);

    %     \path [draw, ->] (vote.south) edge[bend  left=90]  node [below] {Used in validation} 
    %         (syscomb.30);            

    % \end{pgfonlayer}
    % 
    % \path (asr1-2.south) +(-\blockdist,-\blockdist) 
    %     node (asrs) {System Combination - Validation};

  \end{tikzpicture}
\end{figure}